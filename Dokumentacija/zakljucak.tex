\quad Uloga genetskih algoritama u strojnom učenju neosporiva je i značajna u današnjem svijetu. Nevjerojatna je učinkovitost  metode pogreške i pokušaja koja kroz generacije programa stvara najbolja rješenja za različite probleme i okoline u kratkom roku. Fascinantnost ovakvog oblika učenja i razvoja programa uvjetovala je stvaranje ovog rada. 
\par 
Tijekom ovog rada, pomoću Kartezijevog genetskog programiranja, razvijeni su agenti čiji je cilj igranje igre \textit{2048}. Objašnjena su pravila i funkcioniranja verzija igara, objašnjeni su ključni elementi i operacije genetskog programiranja i Kartezijevog genetskog programiranja, izložena je i objašnjena implementacija rada, izvedeni su i prikazani rezultati testova implementacije.
\par 
Na žalost, verzija igre učinila se previše problematičnom za skupljanje veće količine podataka i provođenje više testova. Određeni napreci, nazadovanja i utjecaji parametara mogu se vidjeti, ali za pravi učinak morali bi moći provesti razvoj kroz podosta više generacija nego što je provedeno u ovom radu. 
\par
Implementacija otvara mogućnosti za daljnji rad i istraživanje. Potencijalne modifikacije bile bi olakšavanje unosa i korištenja određenih parametara sustava uz moguće grafičko sučelje koje bi zamijenilo potrebu za unošenjem parametara direktno u programski kod implementacije. Kombinacije različitih parametara gotovo su neograničene i ostavljaju puno prostora za daljnja testiranja. Najveći napredak može se postići dodatnim radom na korištenoj verziji igre, upotrebom druge verzije ili razvojem vlastite kojom bi se ubrzalo vrijeme treniranja i razvoja programa. 