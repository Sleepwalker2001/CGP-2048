Iako je većina igara napravljena za dvoje ili više igrača, razvojem tehnologije, a posebno računala, počinju se razvijati igre za samo jednog igrača. U njima je najčešći cilj dobiti najbolji/najveći rezultat koristeći pravila po kojima igra funkcionira na optimalan način. Tako se često igri pristupa kao slagalici koju u što manje poteza pokušavamo složiti što bolje, a svaki od naših poteza se boduje i utječe na sljedeće poteze.\par
Igra \textit{2048} upravo je takva igra napravljena za jednog igrača. Cilj igre je dobiti što veći rezultat koji je sam po sebi zbroj novostvorenih brojeva na ploči ispred nas. Imamo vrlo jednostavne poteze: pomicanje svih dijelova ploče lijevo, desno, gore ili dolje. Prilikom pomicanja, elementi koji su isti i susjedni, a nalaze se jedan drugome na putanji, spajaju se i time otvaraju prostor na ploči. Nakon svakog poteza se na nasumično otvoreno mjesto stavlja novi element. Igra se čini poprilično jednostavna za shvatiti te ćemo u kasnijem poglavlju vidjeti još neke pojedinosti igre.\par
Unatoč svojoj jednostavnosti, igra može postati podosta kompleksna. Upravljanje praznim prostorom te općenita prostorna organizacija elemenata postaje zahtjevnija kako vrijeme prolazi. Igrač će se prije ili kasnije naći u situaciji kad mu je ploča ispunjena do kraja i više nije u stanju napraviti ijedan potez. U tom trenutku igra se završava.\par 
Cilj ovog rada je stvoriti agenta ili populaciju agenata koristeći Kartezijevo genetsko programiranje koji bi na osnovu stanja ploče napravili što bolji potez u kontekstu cijele partije.\par
U sklopu ovog rada bavit ćemo se konceptima genetskog programiranja, Kartezijevog genetskog programiranja, razvojem agenata kroz generacije, implementacijom specifičnom za ovaj rad, njezinim rezultatima i diskusijom o rezultatima.
